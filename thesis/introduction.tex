\chapter*{Wstęp}
\label{chap:wstep}
\addcontentsline{toc}{chapter}{Wstęp}

Wraz z rozpowszechnieniem się komputerów, powstały różne modele wyszukiwania informacji, jak i ich implementacje. Zostaną one opisane w \textbf{Rozdziale 1}. Techniki te, trwały przez wiele lat, jako główne modele wyszukiwania informacji. Chociażby Okapi BM25 stworzone w latach 80-tych i 90-tych XX wieku jest stosowane do dziś. Zmiana paradygmatu zaczęła w dużej mierze zachodzić dzięki udanemu zastosowaniu technik uczenia maszynowego. Szczególną rolę miał tutaj model uczenia głębokiego \textit{Transformer}, który ostatnimi laty zaczął odgrywać znaczącą rolę w różnych gałęziach uczenia głębokiego. \textbf{Rozdział 2} służy ogólnemu przedstawieniu formalizmów, modeli, zbiorów danych, sposobów trenowania, jak również zgrubnemu opisowi przykładowych możliwych zastosowań. Rozdział ten ma charakter wprowadzający do tematyki. W efekcie niekwestionowanych sukcesów na polu wykorzystania modeli neuronowych, nastąpiła proliferacja badań w kierunku trenowania sieci do celów wyszukiwania informacji. Swoiste Zoo architektur z różnymi charakterystykami, rodzi pytanie o to, które podejście jest najlepsze oraz jakie perspektywy stoją przed tą szybko rozwijającą się dziedziną. \textbf{Rozdział 3} dotyczy dokładnego scharakteryzowania rozważanych w obszarze IR systemów, przedstawienie charakterystycznych typów architektur oraz przedstawieniu ich zalet oraz wad w zakresie efektywności, szybkości działania itd. W rozdziale tym rozważamy też, możliwe rozszerzenie zakresu wyszukiwania informacji o problematykę odpowiadania na pytania (QA), jak również rozważamy systemy agentowe korzystające z uczenia przez wzmacnianie (RL). Ostatecznie w \textbf{Rozdziale 4} prezentujemy system wyszukiwania informacji stworzony przez autora pracy, z wykorzystaniem przedstawionej  wiedzy, którego charakterystyka zostanie również omówiona.