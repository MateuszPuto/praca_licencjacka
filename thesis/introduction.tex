\chapter*{Wstęp}
\label{chap:wstep}
\addcontentsline{toc}{chapter}{Wstęp}

Początki gromadzenia i systematycznego przetwarzania informacji są ściśle związane z pojawieniem się pisma oraz trwałych materiałów piśmienniczych. Jednak aby mówić o procesie wyszukiwania informacji przyszło nam poczekać do powstania antycznych bibliotek. Biblioteka Aleksandryjska, chociaż na pewno nie była pierwszym tworem posiadającym duży zbiór tekstów, wyróżnia się nie tylko swoją częściowo legendarną historią. Dawne podania mówią, że Aleksander Wielki po zobaczeniu biblioteki Ashurbanipala w Niniwie powziął się stworzenia uniwersalnej biblioteki zawierającej wszystkie dzieła podbitych narodów \autocite{phillips2010}. Szacuje się, że kolekcja składała się, w swoim szczytowym okresie, z od 400,000 do 700,000 zwojów, pozyskanych poprzez zakupy, kopiowanie, konfiskaty z przepływających statków czy nawet podstęp jak w przypadku dzieł Ateńskich  \autocite{phillips2010}. Przy tak dużej ilości materiałów źródłowych organizacja przechowywania i wyszukiwania tekstów nabiera niebagatelnego znaczenia. Jednym z patronów takiego systemu jest Zenodot, pierwszy bibliotekarz Wielkiej Biblioteki Aleksandryjskiej. Wprowadził on alfabetyczną organizację dzieł po pierwszej literze imienia autora, co było rozwiązaniem w tamtym czasie niespotykanym \autocite{phillips2010}. To jednak Kallimach z Cyreny, uważany jest za twórcę \textit{pinakes} - 120 zwojowego katalogu autorów i ich dzieł. Informacje o autorze były dodatkowo poszerzone o notę biograficzną, natomiast wpisy o dziełach, zawierały pierwsze słowa utworu, jak i liczbę linijek, które się na niego składały. \autocite{phillips2010}. \newline


Systemy katalogowe były szeroko używane w zastosowaniach bibliotecznych, lecz proces ten nie został zautomatyzowany przez długi okres. Pierwsze rudymentarne techniki wyszukiwania przy użyciu narzędzi pojawiają się na początku XX wieku. Soper wniósł w 1918 r. o patent na przeszukiwanie katalogu przy użyciu kart z dziurkami i światła, które przechodziło przez odpowiednio umieszczone za sobą karty \autocite{sandersoncroft2012}. Goldberg stworzył w latach 20-tych i 30-tych XX wieku urządzenie pozwalające na zautomatyzowane porównywanie strony dokumentu z jego negatywem na rolce filmowej. Kiedy dokładna zgodność została zarejestrowana na fotokomórce, urządzenie wyświetlało zatrzymany mikrofilm \autocite{sandersoncroft2012}. W roku 1950 zbudowano system wyszukiwania oparty o karty perforowane, działający z szybkością 600 kart na minutę. Wtedy też Calvin Mooers, prezentujący pracę naukową, użył jako pierwszy terminu 'wyszukiwanie informacji' (ang. 'information retrieval') \autocite{sandersoncroft2012}. W brytyjskim Royal Society rozważano już jednak dwa lata wcześniej możliwość zastosowania technologii, która wkrótce miała się okazać przełomowa dla wszelkich zadań obliczeniowych. Holmstrom opisywał wtedy komputer UNIVAC, mogący wyszukiwać odniesienia w tekście umieszczonym na taśmie magnetycznej, na podstawie powiązanych z nimi kodów \autocite{sandersoncroft2012}. \newline


W kolejnych latach, wraz z rozpowszechnieniem się komputerów, powstały różne modele wyszukiwania informacji, jak i ich implementacje. Zostaną one opisane w \textbf{Rozdziale 1}. Techniki te, trwały przez wiele lat, jako główne modele wyszukiwania informacji. Chociażby Okapi BM25 stworzone w latach 80-tych i 90-tych XX wieku jest stosowane do dziś. Zmiana paradygmatu zaczęła w dużej mierze zachodzić dzięki udanemu zastosowaniu technik uczenia maszynowego. Szczególną rolę miał tutaj model uczenia głębokiego \textit{Transformer}, który ostatnimi laty zaczął odgrywać znaczącą rolę w różnych gałęziach uczenia głębokiego. \textbf{Rozdział 2} służy ogólnemu przedstawieniu formalizmów, modeli, zbiorów danych, sposobów trenowania, jak również zgrubnemu opisowi przykładowych możliwych zastosowań. Rozdział ten ma charakter wprowadzający do tematyki. W efekcie niekwestionowanych sukcesów na polu wykorzystania modeli neuronowych, nastąpiła proliferacja badań w kierunku trenowania sieci do celów wyszukiwania informacji. Swoiste Zoo architektur z różnymi charakterystykami, rodzi pytanie o to, które podejście jest najlepsze oraz jakie perspektywy stoją przed tą szybko rozwijającą się dziedziną. \textbf{Rozdział 3} dotyczy dokładnego scharakteryzowania rozważanych w obszarze IR systemów, przedstawienie charakterystycznych typów architektur oraz przedstawieniu ich zalet oraz wad w zakresie efektywności, szybkości działania itd. W rozdziale tym rozważamy też, możliwe rozszerzenie zakresu wyszukiwania informacji o problematykę odpowiadania na pytania (QA), jak również rozważamy systemy agentowe korzystające z uczenia przez wzmacnianie (RL). Ostatecznie w \textbf{Rozdziale 4} prezentujemy system wyszukiwania informacji stworzony przez autora pracy, z wykorzystaniem przedstawionej  wiedzy, którego charakterystyka zostanie również omówiona.